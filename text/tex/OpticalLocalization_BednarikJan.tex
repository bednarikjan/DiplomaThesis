%============================================================================
% Jan Bednarik
% E-mail: jan.bednarik AT hotmail.cz
%============================================================================
% encoding: utf8
%============================================================================

%\documentclass[]{fitthesis} % bez zadání - pro začátek práce, aby nebyl problém s překladem
\documentclass[english]{fitthesis} % odevzdani do wisu - odkazy jsou barevné
%\documentclass[zadani,print]{fitthesis} % pro tisk - odkazy jsou černé
%\documentclass[english,print]{fitthesis} % pro tisk - odkazy jsou černé
% * Je-li prace psana v anglickem jazyce, je zapotrebi u tridy pouzit 
%   parametr english nasledovne:
%      \documentclass[english]{fitthesis}

\usepackage[czech,english]{babel}
\usepackage[utf8]{inputenc} %kodovani
\usepackage[T1]{fontenc}
\usepackage{cmap}
\usepackage{url}
\DeclareUrlCommand\url{\def\UrlLeft{<}\def\UrlRight{>} \urlstyle{tt}}

%zde muzeme vlozit vlastni balicky
\usepackage{listings}
\usepackage[toc,page,header]{appendix}
\RequirePackage{titletoc}
\ifczech
  \usepackage{ae}
\fi

%%%%%%% My packages %%%%%%%%

%% Lorem ipsum
\usepackage{blindtext}

%% Captions
\usepackage{caption} % \captionof

%% Acronyms
\usepackage{acro}

%% Tables
\usepackage{booktabs}
\usepackage{tabularx}

\usepackage{siunitx}
\DeclareSIUnit[number-unit-product = {}]{\inch}{''}

\usepackage{xcolor}
\newcommand\todo[1]{\textcolor{red}{[[ #1 ]] \\}}
\newcommand{\vata[1]}{\textcolor{gray}{\blindtext[#1]}}

%% Algorithms
\usepackage[ruled, vlined, linesnumbered, nofillcomment]{algorithm2e}

%%%% Custom commands
% particle
\newcommand{\particle}[2]{(\vec{x_{#1}^{#2}},w_{#1}^{#2})}

\renewcommand{\floatpagefraction}{.9}

% Matrix notation
\newcommand{\matr}[1]{\mathbf{#1}}

%---rm---------------
\renewcommand{\rmdefault}{lmr}%zavede Latin Modern Roman jako rm
%---sf---------------
\renewcommand{\sfdefault}{qhv}%zavede TeX Gyre Heros jako sf
%---tt------------
\renewcommand{\ttdefault}{lmtt}% zavede Latin Modern tt jako tt

% vypne funkci nové šablony, která automaticky nahrazuje uvozovky,
% aby nebyly prováděny nevhodné náhrady v popisech API apod.
\csdoublequotesoff

% =======================================================================
% balíček "hyperref" vytváří klikací odkazy v pdf, pokud tedy použijeme pdflatex
% problém je, že balíček hyperref musí být uveden jako poslední, takže nemůže
% být v šabloně
\ifWis
\ifx\pdfoutput\undefined % nejedeme pod pdflatexem
\else
  \usepackage{color}
  \usepackage[unicode,colorlinks,hyperindex,plainpages=false,pdftex]{hyperref}
  \definecolor{links}{rgb}{0.4,0.5,0}
  \definecolor{anchors}{rgb}{1,0,0}
  \def\AnchorColor{anchors}
  \def\LinkColor{links}
  \def\pdfBorderAttrs{/Border [0 0 0] }  % bez okrajů kolem odkazů
  \pdfcompresslevel=9
\fi
\else % pro tisk budou odkazy, na které se dá klikat, černé
\ifx\pdfoutput\undefined % nejedeme pod pdflatexem
\else
  \usepackage{color}
  \usepackage[unicode,colorlinks,hyperindex,plainpages=false,pdftex,urlcolor=black,linkcolor=black,citecolor=black]{hyperref}
  \definecolor{links}{rgb}{0,0,0}
  \definecolor{anchors}{rgb}{0,0,0}
  \def\AnchorColor{anchors}
  \def\LinkColor{links}
  \def\pdfBorderAttrs{/Border [0 0 0] } % bez okrajů kolem odkazů
  \pdfcompresslevel=9
\fi
\fi

\projectinfo{
	%Prace
	project=DP,            %typ prace BP/SP/DP/DR
	year=2016,             %rok
	date=\today,           %datum odevzdani
	%Nazev prace
	title.cs={Optická lokalizace velmi vzdálených cílů ve vícekamerovém systému},  %nazev prace v cestine
	title.en={Optical Localization of Very Distant Targets in Multicamera Systems}, %nazev prace v anglictine
	%Autor
	author={Jan Bednařík},   %jmeno prijmeni autora
	author.title.p=Bc., %titul pred jmenem (nepovinne)
	%author.title.a=PhD, %titul za jmenem (nepovinne)
	%Ustav
	department=UPGM, % doplnte prislusnou zkratku: UPSY/UIFS/UITS/UPGM
	%Skolitel
	supervisor= Adam Herout, %jmeno prijmeni skolitele
	supervisor.title.p=prof. Ing.,   %titul pred jmenem (nepovinne)
	supervisor.title.a={Ph.D.},    %titul za jmenem (nepovinne)
	%Klicova slova, abstrakty, prohlaseni a podekovani je mozne definovat 
	%bud pomoci nasledujicich parametru nebo pomoci vyhrazenych maker (viz dale)
	%===========================================================================
	%Klicova slova
	keywords.cs={Klíčová slova v českém jazyce.}, %klicova slova v ceskem jazyce
	keywords.en={Klíčová slova v anglickém jazyce.}, %klicova slova v anglickem jazyce
	%Abstract
	abstract.cs={Výtah (abstrakt) práce v českém jazyce.}, % abstrakt v ceskem jazyce
	abstract.en={Výtah (abstrakt) práce v anglickém jazyce.}, % abstrakt v anglickem jazyce
	%Prohlaseni
	declaration={Prohlašuji, že jsem tento semestrální projekt vypracoval samostatně pod vedením pana doc. Ing. Adama Herouta, Phd. Uvedl jsem všechny literární prameny a publikace, ze kterých jsem čerpal.},
	%Podekovani (nepovinne)
	%  acknowledgment={Zde je možné uvést poděkování vedoucímu práce a těm, kteří poskytli odbornou pomoc.} % nepovinne
}

%Abstrakt (cesky, anglicky)
\abstract[cs]{Tato práce popisuje návrh pasivního systému pro optickou lokalizaci leteckých objektů ve vícekamerovém systému schopném autonomně určit globální geografickou pozici sledovaného objektu. Systém staví na využití několika stanic s pozicovatelnými kamerami, přičemž jsou stanice vhodně rozmístěny v daném prostředí. Každá kamerová jednotka provádí detekci a sledování a zasílá data ostatním jednotkám, díky čemuž je možné triangulovat pozici sledovaného objektu. V práci je popsán navržený model kamerové jednotky, rovněž je vysvětlen a proveden proces rektifikace, jehož účelem je redukovat odchylku mezi modelem a reálnou konstrukcí kamerové jednotky. Diskutovány jsou rovněž různé techniky pro vizuální detekci a sledování objektů a jsou vybrány nejvhodnější z nich z hlediska potřeb navrhovaného systému. Dále je popsán návrh, částečná implementace a simulace systému ve fyzikálním simulátoru. V závěru jsou shrnuty další kroky nezbytné pro úspěšný vývoj systému.}
\abstract[en]{This work presents the passive multi-camera optical localization system suitable for autonomous estimation of the global geographical position of the aerial objects. The system utilizes multiple stations with positionable cameras properly placed within the environment. Each camera unit performs detection and tracking and exchanges the information with each other in order to triangulate the target. In the work the proposed model of the camera unit was described and a rectification process reducing the deviation between the model and the construction of the unit was explained and performed. Various visual detection and tracking approaches were discussed and the most suitable ones were proposed for the final solution. The system was designed, partially implemented and simulated in a physical simulator. The next steps necessary for the successful development of the system were eventually summarized.}

%Klicova slova (cesky, anglicky)
\keywords[cs]{optická lokalizace, lokalizace více kamerami, stereovize, detekce objektů, sledování objektů, triangulace, letecké objekty, UAV, fyzikální simuace, ROS, Gazebo, kinematický řetězec, rektifikace kamery}
\keywords[en]{optical localization, mutli camera localization, stereovision, object detection, object tracking, triangulation, aerial objects, UAV, physical simulation, ROS, Gazebo, kinematic chain, camera rectification}

%Prohlaseni
\declaration{Prohlašuji, že jsem tuto bakalářskou práci vypracoval samostatně pod vedením pana prof. Ing. Adama Herouta, Ph.D. Uvedl jsem všechny literární prameny a publikace, ze kterých jsem čerpal.}

%Podekovani (nepovinne)
\acknowledgment{V této sekci je možno uvést poděkování vedoucímu práce a těm, kteří poskytli odbornou pomoc
(externí zadavatel, konzultant, apod.).}

\begin{document}
  % Vysazeni titulnich stran
  % ----------------------------------------------
  \maketitle
  % Obsah
  % ----------------------------------------------
  \tableofcontents
  
  % Seznam obrazku a tabulek (pokud prace obsahuje velke mnozstvi obrazku, tak se to hodi)
\ifczech
  \renewcommand\listfigurename{Seznam obrázků}
\fi
  % \listoffigures
\ifczech
  \renewcommand\listtablename{Seznam tabulek}
\fi
  % \listoftables 

  % Text prace
  % ----------------------------------------------
  %=========================================================================
% (c) Michal Bidlo, Bohuslav Křena, 2008

%=========================================================================
%=========================================================================
\chapter{Introduction}


%=========================================================================
%=========================================================================
\chapter{Related work}


%=========================================================================
%=========================================================================
\chapter{System architecture}


%=========================================================================
%=========================================================================
\chapter{Implementation}


%=========================================================================
%=========================================================================
\chapter{Experiments and results}


%=========================================================================
%=========================================================================
\chapter{Conclusion}


 % viz. content.tex

  % Pouzita literatura
  % ----------------------------------------------
\ifczech
  \makeatletter
  \def\@openbib@code{\addcontentsline{toc}{chapter}{Literatura}}
  \makeatother
  \bibliographystyle{czechiso}
\else 
  \makeatletter
  \def\@openbib@code{\addcontentsline{toc}{chapter}{Literature}}
  \makeatother
  \bibliographystyle{plain}
%  \bibliographystyle{alpha}
\fi
  \begin{flushleft}
  \bibliography{bibliography} % viz. literatura.bib
  \end{flushleft}

  % Prilohy
  % ---------------------------------------------
  \appendix
\ifczech
  \renewcommand{\appendixpagename}{Přílohy}
  \renewcommand{\appendixtocname}{Přílohy}
  \renewcommand{\appendixname}{Příloha}
\fi
  \appendixpage
\ifczech
  \section*{Seznam příloh}
  \addcontentsline{toc}{section}{Seznam příloh}
\else
  \section*{List of Appendices}
  \addcontentsline{toc}{section}{List of Appendices}
\fi
  \startcontents[chapters]
  \printcontents[chapters]{l}{0}{\setcounter{tocdepth}{2}}
  %=========================================================================
%=========================================================================
\chapter{DVD Contents}

%=========================================================================
%=========================================================================
\chapter{Usage of the OLS}
- mozna pridat jeste Building, Running, nebo to dat az do readme.txt

%=========================================================================
%=========================================================================
\chapter{Paper - Excel@FIT 2016}

\includepdf[pages={1,2,3,4,5,6,7,8}, offset=50 -45]{appendix/2016-ExcelFIT-paper.pdf}

%=========================================================================
%=========================================================================
\chapter{Poster - Excel@FIT 2016}

\begin{figure}[htp] 
	\centering {
		\vspace*{-1.5cm}
		\hspace*{-2.2cm}
		\includegraphics[scale=0.32]{appendix/2016-ExcelFIT-poster.pdf}
	}
\end{figure} % viz. appendix.tex
\end{document}
